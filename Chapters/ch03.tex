\chapter{Online Optimization}
This chapter defines online optimization in the context of accelerators. An overview of optimzation algorithms and their classifications is presented. The Robust Conjugate Direction Search algorithm is introduced and some of its applications in the accelerator community are discussed.
\section{Defining Online Optimzation}
Suppose we have a machine in which there is some sort of figure of merit depending on the collective state of relevent components, parts or operation mode. There is no mechanistic, deterministic or probabolistic model for the dependence of the figure of merit based on the relevant components state, but we do know these parameters have effect over our figure of merit. We might as well call these relevant parameters our knobs, since we can use them to tune the figure of merit. The whole system is a black-box. To access diferent values for the figure of merit, i.e., to sample it, we need to change the knobs, and measure the figure of merit.

Now suppose we want to tune the knobs so the figure of merit reches a certain value, or so that it is minimized or maximized. This is an optimization problem, and we might as well call the figure of merit our objective function. If we are able to devise a computer-automated strategy to seek for the desired value or extremum of the objective function, then running this program while the machine is up and working is what we define as online optimization.

The program must read the objective function and current state of the knobs, calculate the desired changes on the knobs, perform the changes evaluate the objective and repeat this process until reaching the desired outcome.

This is exactly where we are when it comes to the Dynamic Aperture. It is a figure of merit related to the nonlinear dynamics, in our case the sextupole magnets. There is no analytical/statistical\footnote{in principle, a surrogate model could be trained to reproduce dynamic aperture given the sextupole strengths as inputs. This is not what we have done so far} model prediciting DA changes given sextupole nudges so we cannot invert the problem and tune sextupoles to the desired DA value. Therefore, online otpimzation fits this problem well.

\section{Robust Conjugate Direction Search}
In the literature, optimization routines and algorithms are usually classified according to whether they rely on the calculation of derivatives (gradient-based) or solely on the comparison of the objective function values (gradient-free). The latter can yet be classified into direct- or indirect-search methods, depending on whether the search of the extremum relies on direct comparisons of the objective function itself or from a mathematical model of it, respectively \cite{numerical_recipes}.

Both gradient-based and gradient-free strategies rely on the comparison of the objective function at different points of the parameter space. If the objective function suffers from noise this can significantly reduce the efficiency of the optimization routine \cite{numerical_recipes, huang2019beam}. In Chap. 7 of Ref.~\cite{huang2019beam}, a review of the most popular optimization algorithms shows how most of them suffer to find minima to, at least, the precision of the noise-$\sigma$ the objective function is subjected to.

The Robust Conjugate Direction Search (RCDS) algorithm is a indirect-search, gradient-free optimization algorithm introduced in Ref.~\cite{Huang:2013}. The algorithm consists of a main loop for constructing and managing optimal search directions along the knobs space (Powell's Method) and a one-dimensional optimizer responsible for a noise-aware search for the minimum along a given direction. The algorithm is capable of optimizing the objective function (find its local maximum/minimum) to at least the precision of the objective-function noise \cite{Huang:2013, huang2019beam}, being thus adequate for online optimization problems. Specifically, for accelerator controls and optimization, the algorithm has been successfully applied to optimize beam steering and optics matching during injection \cite{Huang:2013}, reducing horizontal emittance \cite{Huang:2013, Huang:2015}and optimization of dynamic aperture \cite{Huang:2013,Huang:2015,Liuzzo:IPAC2016-THPMR015,Olsson:IPAC2018-WEPAL047,yang:ipac2022-tupopt064}.
\subsection{Powell's conjugate direction set}

\subsection{Bracetking and line-search}
