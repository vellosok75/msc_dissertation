\chapter{Conclusions}

In this work, the Robust Conjugate Direction search algorithm (\gls*{RCDS}) was studied, implemented in Python and applied to the problem of optimizing the nonlinear dynamics and dynamic aperture of the SIRIUS storage ring.

The experiments focused on using the average injection efficiency (\gls*{ie}) of five injection pulses as the objective function to probe the Dynamic Aperture (\gls*{DA}). While attempts were made using the beam's resilience to dipolar perturbations as the objective function, it did not yield satisfactory results in enhancing the \gls*{ie} in the nominal operation conditions.

The optimization parameters, or knobs, were the sextupole families of the storage ring. The need to optimize dynamics while maintaining constant chromaticity imposed constraints to the selection of families or combinations of families to be used as optimization knobs. Two methods were considered to ensure constant chromaticity: slecting combinations of sextupole families which belong to the null space of the chromaticity jacobian matrix; and selecting a reduced number of free optimization knobs and delegating an online scheme of chromaticity antecipation and correctioin to another set of sextupole families. We also experimented with imposing supplementary symmetries on the nonlinear lattice in the form of additional constraints to reduce the dimensionality of the search space. Optimization proved effective irrespective of the strategy employed to preserve chromaticity and the varying dimensions of the search space.

Optimization was conducted at the machine's nominal working point, denoted as Working Point 1 (\gls*{WP1}), with tunes $(\nu_x, \nu_y)=(49.08, 14.14)$ and two additional working points, \gls*{WP2}, with tunes $(\nu_x, \nu_y)=(49.20, 14.25)$, and \gls*{WP3}, with tunes $(\nu_x, \nu_y)=(49.16, 14.22)$. The optimization in WP1 was extremely successful, yielding a sextupole configuration that achieved the remarkable level of $98\%$ \gls*{ie} in the nominal operational injection conditions. This configuration exhibited no adverse effects on chromaticity and beam lifetime while displaying an increased phase space area and enhanced kick resilience. In \gls*{WP2}, however, the optimization proved more challenging. Despite positive impacts on phase space areas, kick resilience, the resulting efficiency of $79\%$ fell short of operational expectations. In contrast, the optimization in \gls{WP3} was deemed successful. The process not only improved phase space areas and kick resilience but also achieved an \gls*{ie} of $93\%$, with no impact on lifetime and chromaticity.

The primary motivation for exploring different machine operation tunes was to elevate the fractional parts of the tunes, distancing them from integer resonances. This adjustment aimed to mitigate orbit amplification factors, which quantify the sensitivity of the orbit to perturbations. The tunes of $(\nu_x, \nu_y) = (49.16, 14.22)$ of \gls*{WP3}, resulted in improved stability and \gls*{ie} in this WP was made possible due to the online optimization work. Leveraging this new working point and recent enhancements in the orbit feedback systems, the machine achieved a world-record orbit stability, with horizontal orbit variations  of less than $1\%$ of the horizontal beam size, and vertical orbit variations of less than $4\%$ of the vertical beam size, on average.

In conclusion, this study highlights the success of online optimization of nonlinear dynamics using \gls*{RCDS} in 4th-generation storage rings, exemplified by SIRIUS with its large parameter space. The outcomes of this work serve as a benchmark for the effectiveness of the \gls*{RCDS} method and advocate for the inclusion and planning of online optimization routines as a standard tools during the commissioning of future 4th-generation projects.
