\chapter{Conclusions}

In this work, the Robust Conjugate Direction search algorithm (RCDS) was studied, implemented in Python and applied to the problem of optimizing the nonlinear dynamics and dynamic aperture of the SIRIUS storage ring.

The experiments focused on using injection efficiency as the objective function to examine its impact on the Dynamic Aperture (DA). While attempts were made using the beam's resilience to dipolar perturbations as the objective function, it did not yield satisfactory results in enhancing injection efficiency and ensuring repeatability across multiple injection pulses.

The optimization parameters, or knobs, were the sextupole families of the storage ring. The need to optimize dynamics while maintaining constant chromaticity imposed constraints on the selection of families or combinations of families as knobs. Two methods were considered to ensure consistent chromaticity. Additionally, we experimented with imposing supplementary symmetries on the lattice to reduce the dimensionality of the search space. Optimization proved effective irrespective of the strategy employed to preserve chromaticity and the varying dimensions of the search space.

Optimization was conducted at the machine's nominal working point, denoted as Working Point 1 (WP1), with tunes $(\nu_x, \nu_y)=(49.08, 14.14)$ and two additional working points, WP2, with tunes $(\nu_x, \nu_y)=(49.20, 14.25)$, and WP3, with tunes $(\nu_x, \nu_y)=(49.16, 14.22)$. The optimization in WP1 proved successful, yielding a sextupole configuration that achieved a remarkable $98\%$ injection efficiency. This configuration exhibited no adverse effects on chromaticity and beam lifetime, an increased phase-space area and enhanced kick resilience. In WP2, however, the optimization proved more challenging. Despite positive impacts on phase-space areas, kick resilience, and injection efficiency, the resulting efficiency of $79\%$ fell short of operational expectations. In contrast, the optimization in WP3 was deemed successful. The process not only improved phase-space areas and kick resilience but also achieved an injection efficiency of $93\%$, with no impact on lifetime and chromaticity.

The primary motivation for exploring different machine operation tunes was to elevate the fractional parts of the tunes, distancing them from integer resonances. This adjustment aimed to mitigate orbit amplification factors, which quantify the sensitivity of the orbit to perturbations. Notably, in WP3, this approach resulted in improved stability. Leveraging this new working point and recent enhancements in the orbit feedback systems, the machine achieved a world-record orbit stability. Specifically, the horizontal orbit variation is now less than $1\%$ of the horizontal beam size, and the vertical orbit exhibits variations less than $4\%$ of the vertical beam size, on average.

In conclusion, this study highlights the success of online optimization of nonlinear dynamics using RCDS in 4th-generation storage rings, exemplified by SIRIUS with its large parameter space. The outcomes of this work not serve as a benchmark for the effectiveness of the RCDS method and advocate for the inclusion and planning of online optimization routines as a standard tools during the commissioning of future 4th-generation projects.
