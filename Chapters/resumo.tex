% \begin{abstract}[name=Resumo]
%     O acúmulo de feixe no anel de armazenamento do SIRIUS ocorre no esquema de injeção off-axis, o qual depende de uma abertura dinâmica (AD) -- a região contemplando oscilações transversais estáveis -- suficientemente grande. Durante a fase de projeto, a AD do SIRIUS foi numericamente otimizada no modelo acelerador usando diversas técnicas, e durante a fase de comissionamento, a rede magnética ótima foi implementada na máquina. Medidas recentes mostram que a DA do SIRIUS, apesar de suficientemente grande para uma fornecer uma boa eficiência de injeção, pode ainda ser melhorada por meio do ajuste fino das forças dos imãs sextupolos, que governam a dinâmica não-linear e determinam a DA. Adicionalmente, o interesse crescente em operar em diferentes pontos de trabalho da ótica, nos quais a AD geralmente é deteriorada, requer o trabalho de otimização online para alcançar valores de eficiẽncia de injeção aceitaveis para a operação. Neste projeto de mestrao, o estudante estudou e implementou em Python o alogritmo Robust Conjugate Direction Search, e realizou os experimentos de otimização para melhorar a AD e eficiência de injeção no SIRIUS. Os resultados salientam a efetividade do método de otimização online aplicado em anéis de armazenamento de quarta geração.
% \end{abstract}
