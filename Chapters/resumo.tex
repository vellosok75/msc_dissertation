\begin{abstract}[name=Resumo]
    O acúmulo de feixe no anel de armazenamento do SIRIUS ocorre no esquema de injeção \textit{off-axis}, o qual depende de uma abertura dinâmica (AD) -- a região contemplando oscilações transversais estáveis -- suficientemente grande.
    % Durante a fase de projeto, a AD do SIRIUS foi numericamente otimizada no modelo acelerador usando diversas técnicas, e durante a fase de comissionamento, a rede magnética ótima foi implementada na máquina.
     Antes do trabalho de otimização, a AD do SIRIUS se mostrava suficientemente grande para uma fornecer uma boa eficiência de injeção, mas também mostrava sinais de que poderia ainda ser melhorada por meio do ajuste fino das forças dos imãs sextupolos, que determinam a DA.
    %   Adicionalmente, o interesse crescente em operar em diferentes pontos de trabalho da ótica, nos quais a AD geralmente é deteriorada, requer o trabalho de otimização online para alcançar valores de eficiẽncia de injeção aceitaveis para a operação.
      Neste projeto de mestrado, o alogritmo \textit{Robust Conjugate Direction Search} (RCDS) para otimizção online foi aplicado para otimzar a AD e eficiência de injeção no SIRIUS. Os experimentos também foram realizados em outros pontos de operação, com maiores valores para a parte fracionária das sintonias. Aumentar as sintonias é benéfico para a estabilidade órbita, mas geralmente deteriora a AD, o que requer o trabalho de otimização online para alcançar valores de eficiência de injeção que atendam as demandas da operação. Este trabalho mostra como a otimização online foi bem sucedida para otimizar a AD no ponto de operação nominal do SIRIUS, bem como em um novo ponto de operação, onde melhorias na estabilidade de órbita foram observadas. Os resultados salientam a efetividade desse método em anéis de armazenamento de 4º geração.
\end{abstract}
% Beam accumulation into SIRIUS storage ring occurs in a off-axis scheme, on which the efficiency depends on a sufficiently large dynamic aperture (DA): the region comprising stable transverse oscillations.
% % In the design phase, SIRIUS DA was numerically optimized in the accelerator model using various techniques, and during commissioning, the optimized lattice was implemented in the machine.
%  Prior to optmization, SIRIUS DA was sufficiently large for good injection efficiencies, but showed signs it could be further improved upon fine-tuning of sextupole magnet strengths,
% %  which govern the beam nonlinear dynamics
%  which determine  the DA.
% %  The tuning process is known as Online Optimization and has proven succesful in improving the DA and injection efficiency of SIRIUS, in agreement with the observed outcomes in other machines worldwide.
%  In this master's project, the Robust Conjugate Direction Search Algorithm (RCDS) for online optmization was employed to optimize the DA and the injection efficiency at SIRIUS. Online optimization was also carried out in different optics working points, with higher fractional tunes. Higher tunes are beneficial for improving orbit stability, but the different optics usually deteriorates the DA, requiring online optimization work to achieve injection efficiencies meeting operation demands. This work shows how online optimzation was succesful at optimizing the DA in SIRIUS nominal working point as well as in a new working point where orbit stability improvements were observed. The results highlight the effectiveness of this method in 4th-generation storage rings.
