\chapter{Diagnostics tools, measurements processes \& experimental setup}
This is a "methods" chapter. Its first section presents the available beam diagnostics at the storage ring and describes the experimental measurements of relevant quantities such as beam positions, trajectories and orbits, beam current and lifetime, the tunes and chromaticity and how these are dialed at our will during a study. The last two sections discuss the choice of objective functions to probe the Dynamic Aperture and the appropriate selection of sextupole families as the optimzation knobs.
\section{Diagnostics and measurements at the control room}
\subsection{Beam Position Monitors}
To probe the beam's position along the ring, a diagnostic tool consisting on a set of four pick-up antennas placed within the vaccum chamber are used. These are known as Beam-position-monitors (BPMs), and are sketched in the Figure. The antennas are placed in such a manner so the electron beam deposits mirror charges when passing by then and triggers the antenna a certain voltage signal. The determination of the beam displacements is based on the differential signal induced on the antennas when the beam is not at the geometric center, in which case the induced charges are equal. The signal of the antennas is processed in the so-called "Delta/Sigma" scheme, which gives the horizontal and vertical beam displacements according to the following algebra:
\begin{equation}
    x = K_x \frac{(A+D)-(B+C)}{\Sigma}, \quad y = K_y \frac{(A+B)-(C+D)}{\Sigma},
\end{equation}
where $A,B, C,D$ refers to the intensity of the induced signal over the corresponding antenna, $\Sigma=A+B+C+D$ is the sum signal, proportional to the beam's current, and $K_x$ and $K_y$ are calibration factors, which depend  on the BPM geometry  and distances between the antennas. SIRIUS has 160 BPMs distributed along the storage ring. They allow for the determination of the centroid's positions at a turn-by-turn acquisition rate, which is needed for probing of the betatron motion. The signal can also be processes in other acquisition rates, which renders an averaging of the singal and allows to probe information about the orbit.
\missingfigure{BPMS antennas diagram}
\subsection{Beam Current and Injection Efficiency}
Direct-Current Current Transformers (DCCTs) enable the measurement of the stored beam current within accelerator rings (booster or storage ring). A DCCT current monitor works by surrounding the the beam of charged particles in the accelerator ring with a magnetic core. The magnetic field induced by the beam current flowing by the core is then measured, allowing for an accurate determination of the current itself.

 Utilizing the current measurement and the beam revolution period in the respective ring, one can assess the stored charge, and calculate the injection efficiency during storage ring injections. By estimating the charge in the booster or transport line just before injection into the storage ring and the storage ring charge immediately after the injection pulse, is possible to deduce the efficiency of the injection process.The efficiency of the injection can also be estimated from the sum-signal of the BPMs, since it is proportional to the stored current.
\todo[inline]{what is the accuracy and prcision of the current measurements with the DCCT? what are its limitations? what about the sum-signal}

\subsection{Tunes measurement \& control}
When turn-by-turn motion is viewed at a fixed longitudinal position $s$, it consits on the sampling of a harmonic motion. Its fundamental frequency is the tune $\nu$. Any observation of Turn-by-turn (TbT) motion can reveal the tunes upon the appropriate singal processing. For instance, the betatron motion can be Fourier-transformed (discrete Fourier transform via fast-fourier transform algorithm), revealing the BPM signal spectrum. Alternatively the time-domain singal can be fitted to a sinusoid, allowing the determination of the tune as the fundamental frequency.

Precise measurement and online monitoring of the tunes in an accelerator ring can be achieved with the aid of a stripline shaker, which constantly drives the beam with an alternating electric field in a narrow of frquencies, leading to sub-nanometer displacements and inducing small-amplitude, noninterfering with operation betatron motion. The same system also reads back the beam response at that same frequency range. The peak of the beam response signal is identified with the betatron tune.
\missingfigure{FFT of betatron motion and shaker spectrum}

As for changing and manipulating the tunes, formula~\eqref{eq:delta_nu} reveals that changes in the quadrupoles strengths, specially at the quadrupoles at large $\beta$-function sections, allow for the control of the tunes.  Since the tune response to quadrupole strength changes is linear, a tune-response matrix can be constructed, i.e. the Jacobian of the tunes with respect to changes in quadrupoles, so that tune changes can be expressed as
\begin{equation}
    \Delta\boldsymbol{\nu} = \vb{J}_{\boldsymbol{\nu}}\Delta \vb{K},
\end{equation}
where $\Delta\boldsymbol{\nu} = \mqty*[\Delta \nu_x & \Delta \nu_y]^\intercal$ is the tune-shifts vector, $\Delta \vb{K}$ is the vector containing the changes in stregths across all the quadrupole families, and the Jacobian or response matrix has entries
\begin{equation}
    (\vb{J}_{\boldsymbol{\nu}})_{ij} = \pdv{\nu_{i}}{K_j} \approx \frac{\Delta \nu_{i}}{\Delta K_j}, \quad i=x, y,\quad j\in\text{quadrupole families}.
\end{equation}
The system can pseudo-inverted, allowing for the determination of quadrupoles changes required for a desired tune change
\begin{equation}
    \Delta \vb{K} = \vb{J}_{\nu}^{+}\Delta \boldsymbol{\nu}
\end{equation}
where $\vb{J}_{\nu}^{+} = (\vb{J}_{\nu}^{\intercal}\vb{J}_{\nu})^{-1}\vb{J}_{\nu}^{\intercal}$ is the Moore-Penrose pseudoinverse.
\todo[inline]{which families are used when changing tunes? add discussion on chaging the optics when changing tunes}
\subsection{Chromaticity measurements \& control}
Chromaticity characterizes the energy-dependent tune-shift. To measure it, we need to calculate the numerical derivative
$$\xi_u = \pdv{\nu_u}{\delta}\approx \frac{\Delta \nu_u}{\delta},$$
that is, measure the tune-shift $\Delta \nu_u$ induced by the energy-shift $\delta$. A direct manner to induce a particular energy-shift is to change the RF cavities frequency (see anexxes for brief overview of longitudinal dynamics and details about momentum compaction). A relation can be established between energy deviations $\delta$ and relative RF frequency changes with the aid of a quantity $\alpha$, known as \textit{momentum compaction factor} \cite{lee_accelerator_2004,sands_physics_1969}:
\begin{equation}
    \delta = -\frac{1}{\alpha}\frac{\Delta f}{f}.
\end{equation}
The momentum compaction factor relates changes in orbit length with energy deviations and is defined by
\begin{equation}
    \alpha = \frac{1}{L}\oint G(s) \eta(s) \dd{s}.
\end{equation}
Therefore, in practice, when measuring chromaticity we are interested in the numerical derivative
\begin{equation}
\xi_u = -\frac{f}{\alpha}\frac{\Delta \nu_u}{\Delta f}
\end{equation}
which is obtained as the first-degree coefficient (properly normalized by $\alpha/f$) of the polynomial fitting of the tune-shift vs. RF frequency curve. A typical tune-shift curve is shown by Fig.
\missingfigure{energy tune shift and chromaticity meas}

As eq.~\eqref{eq:chromaticity} shows, chromaticity depends linearly on the chromatic sextupole families strengths. It can thus be dialed to certain desired values according to the same pseudo-inversion procedure described above for the tunes. We relate the chromaticity changes $\Delta \boldsymbol{\xi}\in\mathbb{R}^2$ to the sextupole families strength changes $\Delta \vb{S}\in\mathbb{R}^{d_s}$ by
\begin{equation}
    \Delta\boldsymbol{\xi} = \vb{J}_{\boldsymbol{\xi}}\Delta \vb{S},
\end{equation}
where $\Delta\boldsymbol{\xi} = \mqty*[\Delta \xi_x & \Delta \xi_y]^\intercal$ and the Jacobian matrix $\vb{J}\in\mathbb{R}^{2 \times d_s}$ has entries
\begin{equation}
    (\vb{J}_{\boldsymbol{\xi}})_{ij} = \pdv{\xi_{i}}{S_j} \approx \frac{\Delta \xi_{i}}{\Delta S_j},\quad i=x, y, \quad j\in\text{sextupole families}.
\end{equation}
$d_s$ refers to cardinality of the set of sextupole families used in the chromaticity change process. In principle, at least two families are required for correcting/tuning chromaticity in the machine: one family for each plane. Since the chromatic sextupole families are the only ones effectively changing chromaticity to leading order, then, at most, $d_s=15$.

If we wish to change chromaticity by a $\Delta\boldsymbol{\xi}$ amount, the jacobian can be pseudo-inverted to calculate the required sextupole strength changes:
\begin{equation}
    \Delta \vb{S} = \vb{J}_{\xi}^{+}\Delta \boldsymbol{\xi}.
\end{equation}

In practice the chromaticity jacobian was never actually measured in the real machine, due to the time-consuming process of varying a single sextupole family, carrying out the chromaticity measurement and repeating the processs for the 15 chromatic families. The "measurement" is instead carried out in the SIRIUS storage ring computer model. The model-calculated jacobian renders a satisfactory correction or tuning of the chromaticity in the actual machine.
\todo[inline]{which families are used for correctiing}
\section{The choice of objective function}
There is no analytical formula for relating the storage ring linear or nonlinear optics to the Dynamic Aperture. The optimization procedure must be a direct search procedure: changes are performed in the knobs and the effect over dynamic aperture is evaluated.

Also, we cannot measure dynamic aperture directly. We must choose an objective function to act as a probe to the DA: a figure of merit related to the dynamic aperture to represent it.

Two objectives usually adopted as probes are the injection efficiency and the beam's resilience to dipolar perturbations. The former is quite self-explanatory: the larger the dynamic aperture, the larger space for the beam to be captured during injection, and thus the larger the injection efficiency. The latter is related to the DA by the following: the larger the horizontal dipolar kicks the beam can survive, the larger the orbit distortions towards the positive or negative horizontal plane (depending on the kick direction). So the larger the amplitudes the beam explores as it oscillates, probing the DA borders. If the beam survives to large kicks, it means the ring can accomodate larger orbit distortions because of an increased dynamic aperture
.

In summary, the dynamic aperture optimization procedure must consist on the exploration of sextupole (knobs) configurations yielding the largest dynamic aperture as accessed by as objective function such as injection efficiency or beam kick-resiliency.

\subsection{Injection scheme for acummulation at SIRIUS storage ring}
Beam acummulation into the storage ring occurs in the off-axis scheme. The beam is delivered at $x\approx-8~\unit{mm}$, and receives the kick from the nonlinear kicker field. The field profile is nonlinear, with zero field and gradient at the center of the axis, so that it does not disturbs the stored beam.
In the off-axis scheme, a sufficiently large dynamic aperture is desired to allow the beam to be captured into the storage ring. The predicted efficiency for SIRIUS setup, considering a dynamic aperture reaching $x=-9~\unit{mm}$, was nearly $100\%$. What was observed during 2022 was an injection efficiency of about $88\pm8\%$.

\section{Selection of optimization knobs}

The dynamic aperture is determined by the quality of the dynamics in terms of perturbations and nonlinearities. Considering corrected quadrupoles and dipoles (linear optics), the main factors influencing SIRIUS DA are the nonlinearities introduced by the sextupoles and possbibly their field's small errors and deviations from the design parameters. The goal, thus, is to search for the sextupole configurations rendering the largest DA.

The sextupoles are the parmeters which can be tuned, the knobs. SIRIUS has 21 sextupole families: magnets powered by the same power supply. 6 of them are achromatic sextupoles. They are placed where the the dispersion is zero. The 15 other families are chromatic families. Table~\ref{fams} shows the 21 sextupole families names.
 \begin{table}[htb]
    \caption{SIRIUS sextupole families}
    \centering
    \begin{tabular}{cl}
            \hline
            achromatic & \begin{tabular}[c]{@{}l@{}}SFA0, SDA0, \\ SFB0, SDB0,\\ SDP0, SFP0\end{tabular}                                                                \\ \hline
            chromatic  & \begin{tabular}[c]{@{}l@{}}SDA1, SFA1, \\ SDA2, SFA2, \\ SDA3,\\ SDB1, {SFB1}\\ SDB2, SFB2, \\ SDB3, \\ {SFP1}, SDP1,\\ SDP2, SFP2\\ SDP3\end{tabular} \\ \hline
            \end{tabular}
            \label{fams}
            \end{table}
In principle, thus, the optimization parameter space is 21-dimensional. In reality, we would like to change sextupoles without changing chromaticity. Since we need at least one degree of freedom for correcting chromaticity in the horizontal plane and one degree of freedom for correcting the chromaticity in vertical plane, there are 19 available knobs. The dimensionality of the search space can be further reduced by imposing additional constraints to certain families variations. The specific choices of knobs for optimzation experiments are discussed in more details in the Results section.


\subsection{Characterization of Sextupole Magnets Configurations}
Once a configuration of sextupoles (position in parameter space) is found, the nonlinear optics it provides the machine needs to be characterized. The characterizations consisted on evaluatig/measuring the followint figures of merit and desired features
\begin{itemize}
    \item Injection efficiency in nominal off-axis conditions : this is the most desired characteristic. The sextupoles are to be optimized so the DA and the off-axis injection efficiency increase.
    \item Beam Kick resilience: a small current of $2~\unit{\milli\ampere}$, concentrated in a single bucket is stored in the ring. The beam is kicked by the horizontal dipole kicker, which instantly provides a dipolar perturbation leading the beam to be displaced in the horizontal direction. The current before and after the kick is recorded by a current monitor (DCCT) and allows for the calculation of the fraction of the beam lost as a consequence of the kick and the transverse displacement. The procedure is repeated with progressively stronger kicks, and a curve of beam loss as a function of the kick can be constructed. The smaller the losses for larger kicks, the larger the resilience.
    \item Phase portrait area: it is expected that the optimzation procedure increases the dynamic aperture of the machine, meaning it can accomodate larger oscillations and larger phase portraits $x-x^\prime$. Using beam position monitors (BPMs) at the two ends of a straight section, which record the positions of the beam centroid at each turn, we can calculate the position and angle of the beam in the middle of the straight section, and thus recostruct the phase-portrait from turn-by-turn (TbT) data.
    \item Beam lifetime: the lifetime at SIRIUS is dominated by losses due to electron-electron interactions leading to momentum transfers exceeding the energy/momentum acceptance (MA). Optimization of DA does not necessarily leads to improvements in the MA. If the MA is reduced, the rate at which the beam is lost can increase, reducing the total lifetime. It is desireble that the configurations found during DA otpimzation do not worsen the MA and beam lifetime considerably.
    % One dominat effect leading to beam loss is the so-called Touschek Effect, consisiting on collisions between electrons of the same bunch leading to momentum trasfers from the transverse to the longitudinal direction that can exceed the lattice tolerance to momentum deviations, the Momentum Acceptance. Momentum acceptance is the dynamic aperture for off-energy particles. Optimization of DA, which is accesed by the aforementioned objective functions, not necessarily leads to improvements in the MA. If the MA is reduced, the rate of Touschek events increases and thus the lifetime is reduced. Therefore, another desired feature from a good sextupole configurations is having a good beam lifetime. We expect not the worsen the lifetime.
    \item Chromaticity: Sextupoles are introduced in the storage ring for correction of focusing chromatic aberretions. When changing the sextupole settings, it is desired to do so in such a manner that the chromaticity is unchanged. The methods for choosing the optimization knobs already take into account the need for keeping constant chromaticity. Still, after optimization is performed, we need to check whether chromaticity is unchanged.
\end{itemize}
The first two characterizations are quite similar to the two most immediate objective function candidates mentioned above. Indeed, in most nonlinear dynamics optimzation experiments, optimization using injection efficiency or kick resilience as objectives seemed to be completely interchangable. Improvements in injection efficiency necessarily led to improvements in kick resilience, and vice-versa. As shown in more details in the results section, for the SIRIUS storage ring this appears not to be the case. The configurations can be specialized to improvements solely on injection efficiency or solely to kick resilience. This feature was observed during the characterization of the optimized sextupole settings with respect to these two figures of merit.
