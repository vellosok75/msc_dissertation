\chapter{Experiments with Online Optimzation of Nonlinear Dynamics}
This chapter presents the results of the experiments carried out during the this masters project. There are two set of important results: i) those of the experiments carried out in late 2022 and ii) those of the experiments carried out during the first semester of 2023. In i), the early attempts, the experimental method and setup was still being perfected. We tried using the beam kick resilience as objective function and learned it was not the best choice. In ii), we moved on to using instead the injection efficiency as objective, which had a better performance.  We started to take more care when choosing the knobs, avoiding the families close to their magnetic nonlinear regime and experimenting with different possibilities of constraints among the sextupole families. We also carried out optimization in different working tunes and performed more detailed characterizations and analysis of the found configurations.
\section{Kick resilience optimization attempt}
In the first attempt to online optimize the nonlinear dynamics, we used the beam kick resilience as objective function. As described in subsection~\ref{subsec:objective_function}, we sought to minimize the beam-loss rate at a given fixed dipolar kick from the pinger magnet. The idea was to minimize the rate for a given kick, increase the kick and repeat the process, reaching higher kicks.
\subsection{The knobs}
The knobs were chosen according to the compensation scheme described in subsection~\ref{subsubsec:compensation}: the achromatic families SDA0, SDB0, SDP0, SFA0, SFB0, SFP0, and the chromatic families SDA1, SDB1, SDP1, SDA3, SDB3, SDP3, SFA1, SFB1, SFP1 varied freely. The SDA2, SDB2, SDP2 and SFA2, SFB2, SFP2 families were the compensation families used to keep chromaticity constant when varying the optimization knobs.
\subsection{Objective function and setup}
A small beam current was accumulated into the storage ring, usually $10~\unit{mA}$, localized in a single bunch. At a given moment, we can fire up the kick from a dipole kicker, located close to the injection section. The BPMs acquisition was fired in synchrony with the kick. Since we are interested in optimizing the horizontal DA, the kick was in the horizontal direction. The scheme below sketches the $x,x^\prime$ phase space during the experiment. For small kicks, in the linear approximation, the beam would receive an action-jump along the $x^\prime$ axis and start to rotate along the corresponding ellipse. In the nonlinear regime, the ellipses are distorted, but we expect the same overall behaviour: action jump along the angles which is eventually translated into horizontal oscillations. Thus, the larger the kick resilience, larger the DA.
\begin{figure}
    \centering
    \includegraphics[width=\textwidth]{Images/phase_space_kick.pdf}
    \caption[Action jumps due to dipolar kicks in the linear and nonlinear regimes.]{Action jumps due to dipolar kicks in the linear (left) and nonlinear (right) regimes.}
\end{figure}
% \todo[inline]{move the phase-space discussion to the previous chapter as well as figure illustrating it}
To evaluate the beam loss we used the BPMs sum signal, which is proportional to the stored current. The average sum-signal of the first 10 turns was compared to that of the last 10 turns the BPMs acquisition time-series. The kick strength was chosen to render an initial beam rate loss of about 35\% up to 60\%
\subsection{Optimization runs \& results}
With the aforementioned scheme for changing strengths in the sextupole families, RCDS was started to minimize the beam-loss upon the horizontal kick. In the algorithm's first iteration\footnote{An RCDS iteration is reached upon completing the one-dimensional optimization along all directions in the parameter space. After each iteration, the algorithm constructs a new (conjugate) direction according to Powell's method and may replace existing directions by this new conjugate direction.}, beam loss dropped from 60\% to nearly 0\%. In the beginning of the 2nd iteration, the objective function took negative values, which is a numerical artifact, so the optimization run was stopped. The beam-loss minimization significantly improved the beam's resiliency to dipole kicks. After the optimization, it was necessary to kick the beam at approximately $\Delta x^\prime=-0.850~\unit{m rad}$ to achieve the same  30--60\% beam-loss rate previously achieved by a $\Delta x^\prime=-0.760~\unit{m rad}$ kick.

By the end of this first attempt, the machine magnets were standardized\footnote{Standardizing magnets consists on driving their power suplies with decaying sinusoidal waveform to remove hysteresis effects and bring the magnets yokes to their standard reference magnetization.} and the configurations found during optimization were loaded into the machine sextupoles. This was done to test the repeatability of the configuration found. Given the improvements in the resilience, it was expected the injection efficiency would also improve as a result of the DA enlargement, however, when trying to inject in the off-axis scheme, the efficiency was quite low, indicating no DA improvements in the $-x$ direction at all. The improved kick resilience, however, was preserved. This observation raised the suspicion that the aperture along the negative horizontal direction might have been negatively impacted by the procedure, while the aperture along $x^\prime$ increased. In other words, the optimization was not evenly distributed along both $x$ and $x^\prime$. The DA border apparently is way more elastic than anticipated, and apparently can be stretched preferentially along $x$ or $x^\prime$, as the Figure~\ref{fig:expected_vs_reality_DA} illustrates. This observation motivated the adoption of injection efficiency to probe the DA.
\begin{figure}
    \centering
    \includegraphics[width=\textwidth]{Images/elastic_phase_space_distortion.pdf}
    \caption[Expected phase-space ellipse distortions vs. hypothetically realized distortions.]{Expected phase-space ellipse distortions, in the left. Hypothetically realized distortions, preferentially along the $x^\prime$ axis.}
    \label{fig:expected_vs_reality_DA}
\end{figure}
\section{Injection efficiency optimization}
The initial attempt to optimize the Dynamic Aperture (DA) by minimizing beam loss revealed that the optimization procedure did not enhance injection efficiency, suggesting no impact on the DA. This led to the decision to use the injection efficiency itself as the objective function. Changes are made to the sextupoles, and each evaluation of the objective function involves the average of 5 injection pulses into the storage ring. With this configuration, the error sigma of the objective function was approximately $\sigma=1\%$. The concept behind optimizing injection efficiency is to modify the injection conditions in a way that reduces efficiency, with the subsequent increase being a result of enlarging the DA.

Extra attention was given to the injection conditions and the anticipated beam positions during this process, taking into account the seemingly elastic nature of the Dynamic Aperture (DA) boundary. The off-axis injection efficiency was intentionally reduced by lowering the NLK kick strength, placing the beam slightly above the nominal $x^\prime\approx 0$. In practice, a value of $x^\prime\approx 0.100~\unit{mrad}$ was typically set in the experiments. Consequently, the beam was injected at the upper-left border of the $(x,x^\prime)$ aperture, as illustrated in Figure~\ref{fig:inj_cond}. The efficiency under such conditions was approximately $30\%$. The expectation was that maximizing injection efficiency in these conditions would correspond to a more even enlargement of the DA in both the $x$ and $x^{\prime}$ directions, stretching the boundary diagonally in the upper-left quadrant. This is in contrast to the previous attempt, where the enlargement seemed to occur preferentially along the $x^\prime$ direction. Once the procedure is complete, the DA is anticipated to be larger than in the initial state, and it is expected that injection under nominal conditions $(x, x^\prime)\approx(-8.5~\unit{mm}, 0)$ would be significantly more efficient.
\begin{figure}[b]
    \centering
    \includegraphics[width=0.7\columnwidth]{Images/inj_cond.png}
    \caption[Injection conditions for DA optimization.]{Injection conditions for DA optimization.}
    \label{fig:inj_cond}
\end{figure}

Moreover, a significant departure from the early attempt was the exclusion of families SFP1 and SFB1 as knobs in the optimization experiments. This decision was made because these families operate near their saturation strengths, where hysteresis effects become prominent, as discussed in Section~\ref{subsubsec:compensation}. The optimization experiments were conducted in the machine with the nominal tunes $(\nu_x,\nu_y)=(49.08, 14.14)$, referred to as Working Point 1 (WP1), as well as in the tunes $(49.20, 14.25)$ and $(49.16, 14.22)$, denoted as Working Points 2 (WP2) and 3 (WP3), respectively. As mentioned earlier, the goal was to explore a different optics configuration with smaller orbit amplification factors to enhance orbit stability.
% The Results reported here have also been presented in Ref.\cite{velloso:ipac23-wepl087} (on WPs 1 and 2) and on the presentation  delivered at the Optics Tuning and Corrections for Future Colliders Workshop.
\subsection{Optimization in Working Point 1 (49.08, 14.14)}
The knob selection scheme followed the Constrained Scheme I, as detailed in Section~\ref{subsubsec:nullspace}. Three optimization experiments were conducted, resulting in three optimized configurations. In Run 1, the optimization began with the reference sextupole configurations. The machine underwent linear optics and orbit corrections before initiating the optimization process. The optimization was started and once the best injection efficiency was achieved, the run was halted and the best-performing sextupole configuration was saved. These optimal configurations are referred to as "solutions". The magnets were standardized, and the solution from Run 1 was implemented in the machine. Run 2 commenced with the solution from Run 1. Since the Run 1 solution improved the efficiency, the horizontal offset was further increased during injection to reduce the efficiency by shifting the beam toward the border of the expectedly enlarged DA. The same procedure was replicated for Run 3, which initiated from the solution obtained in Run 2. The figure below illustrates the history of the objective function throughout the optimization runs at Working Point 1.
\begin{figure}[tb]
    \centering
    \includegraphics[width=\columnwidth]{Images/wp1_objfunc_hist.pdf}
    \caption[Objective function history along the RCDS evaluations in WP1.]{Objective function history along the RCDS evaluations in WP1.}
\end{figure}
\subsubsection{Characterization of solutions}
For each of the optimal configurations identified in Runs 1, 2, and 3, as well as for the non-optimized reference configuration (ref. config.), turn-by-turn (TbT) BPM data of the stored beam subjected to horizontal dipolar kicker kicks was collected. The DCCT current monitor allowed the determination of the current losses as a function of the horizontal kicks, which is shown in Figure~\ref{fig:loss_kicks} for the three runs. These curves characterize the beam's resilience to kicks.

TbT data also allowed for the reconstruction of the $(x,x^\prime)$ phase space of the beam under the influence of the kicks. Using two BPMs at the ends of an empty ID straight section, the position and angle of the beam were determined at each turn. Figure~\ref{fig:oldtunes_phase} shows the measured phase spaces for the ref. config. and the best configurations found during run 1, 2, and 3, at storage ring fifth straight section (SA05), which is a high-beta section with optics identical to that of the injection point. In the measurement, the beam was under the influence of kicks rendering approximately the same current loss of $12\%$.
\begin{figure}[tb]
    \centering
    \includegraphics[width=0.6\columnwidth]{Images/WEPL087_f1.pdf}
    \caption[Current losses vs. horizontal dipole kick for the ref. config. and for the RCDS solutions at WP 1.]{Current losses vs. horizontal dipole kick for the ref. config. and for the RCDS solutions at WP 1.}
       \label{fig:loss_kicks}
\end{figure}
\begin{figure}[tb]
    \centering
        \includegraphics[width=\textwidth]{Images/WEPL087_f2.pdf}
        \caption[Measured phase space at SA05 high-beta straight section for the ref. config. and the best RCDS configurations of runs 1, 2 and 3 in WP 1.]{Measured phase space at SA05 high-beta straight section for the ref. config. and the best RCDS configurations of runs 1, 2 and 3 in WP 1. Color-map indicates the turns. The areas are in $\unit{mm}~\unit{mrad}$. The beam was being kicked horizontally at $730~\unit{\micro rad}$ in the ref. config, $790~\unit{\micro rad}$ in run 1, $780~\unit{\micro rad}$ in run 2, and $770~\unit{\micro rad}$, in run 3. Loss rates of  $12\%, 11\%, 13\%$ and $13\%$.}
        \label{fig:oldtunes_phase}
\end{figure}

Table~\ref{table1} compiles the injection efficiencies achieved for each configuration during off-axis NLK injection under normal injection conditions ($x\approx -8.5~\unit{mm}$, $x^\prime\approx 0 $).
Once again, we emphasize the apparent elasticity of the phase portrait ellipse deformations: the configuration with the highest kick resilience, observed in Run 1, is not necessarily the one with the largest phase space area and best injection efficiency performance. This behavior could be explained if the phase space deformations of the ellipse at the kicker location for this sextupole setting resulted in a larger $x^\prime/x$ ratio, contributing to a greater kick acceptance and poorer injection performance compared to Run 2. In summary, increased kick resilience does not necessarily translate into an increased DA.

Lifetime at $60~\unit{mA}$ was measured at $20~\unit{hr}$ for run 2 solution, the best performing in terms of injection efficiency. The measurement revealed no impact of the optimization procedure on lifetime, since lifetime for the same conditions on the reference configuration is $21~\unit{hr}$.

Despite the precautions taken to avoid changing chromaticity during the procedure by selecting the chromaticity Jacobian null space knobs, a slight build-up was observed. Chromaticity was measured as $(2.33, 2.53)$ in the reference configuration and $(2.24, 2.39)$ in the solution obtained in Run 2. This could be attributed to minor discrepancies between the machine computer model and the actual machine, as the optimization knobs were computed using the storage ring computer model Jacobian. Despite the small changes in chromaticity, the observed values still fall within acceptable ranges according to criteria related to impedance budgets.
\todo[inline]{clarify this}

In summary, for WP1:
\begin{itemize}
    \item the solution found during run 2 rendered $98\%$ injection efficiency,
    \item increase in horizontal phase space area and horizontal kick resilience were observed,
    \item no significant effect was observed on beam lifetime,
    \item small, acceptable chromaticity changes observed.
\end{itemize}
The first two items are strong indicators of a DA enlargement.
\begin{table}[tb]
    \caption{Injection efficiencies (IE) for configurations found for Working Points 1, 2 and 3.}
    \centering
    \begin{tabular}{cccccc}
    \hline
    \multicolumn{2}{c}{working point 1} & \multicolumn{2}{c}{working point 2}         & \multicolumn{2}{c}{working point 3}         \\ \hline
    configuration      & IE $[\%]$      & configuration        & IE $[\%]$            & configuration        & IE $[\%]$            \\ \hline
    ref. config.       & $88\pm8$       & initial              & $51\pm1$             & initial              &                      \\
    run 1              & $91\pm1$       & run 1                & $79\pm3$             & optimized            & $93\pm3$             \\
    run 2              & $98\pm1$       & run 2                & $65\pm1    $         &                      &                      \\
    run 3              & $87\pm3$       & \multicolumn{1}{l}{} & \multicolumn{1}{l}{} & \multicolumn{1}{l}{} & \multicolumn{1}{l}{} \\ \hline
    \end{tabular}
    \label{table1}
    \end{table}

\subsection{Optimization in Working Point 2 (49.20, 14.25)}
The storage ring tunes were adjusted to $(\nu_x, \nu_y)=(49.20, 14.25)$ using the tune quadrupole knobs, as discussed in section X. The sextupole configuration was the same as the nominal tunes reference configuration. The new optics significantly impacted on the DA, since the injection efficiency in nominal conditions with this setup was about $50\%$, at most. Without successful optimization, it would be impossible to operate in this working point.

For the optimization experiment, the objective function was the injection efficiency with the the beam delivered at the upper-left border of the $x,x^\prime$ phase-spacek, just as in the WP2 experiment. The optimization knobs were those of the Constraints Scheme II, discussed in \ref{subsec:knobs}, totalling 17 independent knobs. In working point 2, two optimization runs were carried out: run 1 and run 2. The sextupoles were optimized from the new tunes optics with reference sextupoles and then, from the best solution found in run 1, run 2 was launched. The objective function history throughout the optimization is shown in Figure
\begin{figure}
    \includegraphics[width=\columnwidth]{Images/wp2_objfunc_hist.pdf}
    \caption[Objective function history along the RCDS evaluations in WP2.]{Objective function history along the RCDS evaluations in WP2.}
\end{figure}
\subsubsection{Characterization of solutions}
Injection efficiency in nominal conditions is highlighted in Table~\ref{table1}. Despite observing improvements, the best performing configuration, the solution found in run 1, still provides an unsatisfactory efficiency for operation.

TbT BPM data of the kicked stored beam  in the initial configuration (non-optimized) and in each run's best solution was acquired and allowed the determination of current losses vs. kicks, shown in Fig.~\ref{fig:loss_kicks_newtunes}, and the reconstruction of phase space, shown in Fig. ~\ref{fig:newtunes_phase}. Improvements on the resilience and the phase-space area can be observed.
\begin{figure}
    \centering
    \includegraphics[width=0.7\columnwidth]{Images/WEPL087_f3.pdf}
    \caption[Current losses vs. horizontal dipole kick for the initial configuration and the RCDS solutions at WP 2.]{Current losses vs. horizontal dipole kick for the initial configuration and the RCDS solutions at WP 2.}
    \label{fig:loss_kicks_newtunes}
\end{figure}
\begin{figure}[tb]
    \includegraphics[width=\textwidth]{Images/WEPL087_f4.pdf}
    \caption[Measured phase space at SA05 high-beta straight section for the initial configuration and the best RCDS configurations of runs 1 and 2 in WP 2.]{Measured phase space at SA05 high-beta straight section for the initial configuration and the best RCDS configurations of runs 1 and 2 in WP 2. Color-map indicates the turns. The areas are in $\unit{mm}~\unit{mrad}$. The beam was being kicked horizontally at $680~\unit{\micro rad}$, for the initial configuration, $770~\unit{\micro rad}$ for run 1, and at $720~\unit{\micro rad}$ for run 2. Loss rates of $10\%$, $12\%$ and $12\%$, respectively}
    \label{fig:newtunes_phase}
\end{figure}

The configuration found during run 1 rendered the best injection efficiency, the largest kick resilience. It also displayed larger lifetime than the initial configuration ($21~\unit{hrs}$, run 1 vs. $18~\unit{hrs}$, initial, at $60~\unit{mA}$) which is comparable to the reference configuration lifetime. The largest phase-space area increase was also observed for this solution. Still, the injection efficiency delivered by the best performing solution on this working point was quite unsatisfactory and optimizing in this working point was hard, as the objective function history shows. The DA seemed more rigid. For these reasons, another working point was sought. If the idea is to increase the fraction parts of the tunes, and $(0.20, 0.25)$ seemed like overshooting, optimization in the intermediate tunes between WP1 and WP2, with fractional tunes $(0.16, 0.22)$, seemed reasonable.
\subsection{Optimization in Working Point 3 (49.16, 14.22)}
From the reference configuration with corrected linear optics and orbit, the tunes were adjusted to the desired working point $(\nu_x, \nu_y)=(49.16, 13.22)$. The injection efficiency was again lower, indicating, as in WP2, deterioration of the DA.

The objective function was injection efficiency in the same conditions as in WP1 and WP2 experiments. The optimization knobs were those of the Compensation Scheme described in section \ref{subsec:knobs}. The search space was 13-dimensional. Two optimization runs were carried out, starting from sextupole settings of the reference configuration. Best configuration found at run 1 was reloaded after magnets standardization and run 2 was launched. Only run 2 configuration was saved. The figure shows the objective function history.
\begin{figure}[tb]
    \centering
    \includegraphics[width=\columnwidth]{Images/wp3_objfunc_hist.pdf}
    \caption[Objective function history along the RCDS evaluations in WP3.]{Objective function history along the RCDS evaluations in WP3.}
\end{figure}
\subsubsection{Characterization of the solution}
The optimized solution was characterized in terms of kick resilience, phase space area, injection efficiency and whether it preserved chromaticity and lifetime.  Lifetime at $60~\unit{mA}$ was measured at $19.5~\unit{hrs}$, so no significant reductions were observed. No significant chromaticity changes were observed as well. Phase space area increased, compared to the initial non-optimized configuration, and it reached similar area to that of the nominal tunes reference configuration, as Figure shows. Kick resilience, shown in Figure, also improved, with a larger fraction of the beam surviving to large kicks in the range from $700-770~\unit{\mu rad}$. Most importantly, run 2 solution displayed injection efficiency of $93\pm3\%$ during nominal off-axis injection, which is acceptable for operation.
\begin{figure}[tb]
    \centering
    \includegraphics[width=0.7\columnwidth]{Images/wp3_kick_resilience.pdf}
    \caption[Current losses vs. horizontal dipole kick for the initial configuration and the RCDS solution at WP 3.]{Current losses vs. horizontal dipole kick for the initial configuration and the RCDS solution at WP 3.}
    \hfill
\end{figure}
\begin{figure}[tb]
    \centering
    \includegraphics[width=\textwidth]{Images/wp3_phase_space.pdf}
    \caption[Measured phase space at SA05 high-beta straight section for the non-optimized configuration and the best RCDS configuration in WP 3.]{Measured phase space at SA05 high-beta straight section for the non-optimized configuration and the best RCDS configuration in WP 3. Color-map indicates the turns. The areas are in $\unit{mm}~\unit{mrad}$.}
\end{figure}
\subsubsection{Orbit Stability}
Orbit stability improvements were confirmed by the orbit integrated spectrum density, which decreased by a factor of approximately 2 \cite{liu_status_2023}, as Fig.~\ref{fig:integrated_spec} shows. Orbit rms variations reached the record values of less than $1\%$ of the horizontal beam size, in the horizontal plane, and less than $4\%$ of the vertical beam size in the vertical plane.

% In summary, in the experiments throughout 2023, the noise in the objective function was reduced and the injection efficiency average was established as the standard objective. No significant chromaticity changes were observed during/after the optimization runs, nor significant changes to beam lifetime, compared to the nominal working point reference configuration.

% Excellent configurations were found in WP1, with 98\% injection efficiency, but we still believe there is room for further improvements in the higher tunes configurations.
\begin{figure}[tb]
    \centering
    \includegraphics[width=\textwidth]{Images/WEOGA2_f5.png}
    \caption[Horizontal/Vertical RMS orbit variations in units of the horizontal and vertical beam sizes.]{Horizontal/Vertical RMS orbit variations in units of the horizontal and vertical beam sizes. Blue curves represents variations in the nominal working point, WP1, orange curves are the orbit variations at WP3, and green curves variations at WP3 plus results of the recent improvements in Fast Orbit Feedback System. From~\cite{liu_status_2023}}
    \label{fig:integrated_spec}
\end{figure}

\subsection{Amplitude-dependent tune-shift analysis}
Online optimization is a heuristic optimization approach. The sextupole configurations found can be wildly different among themselves and from the nominal sextupole lattice strengths. One may wonder if any the RCDS-optimized sextupole strengths share any common characteristic.

One intrinsic feature of nonlinear dynamics and relatively easy to experimentally access is the nonlinear tune-shifts. Specifically, the transverse amplitudes tune-shifts. By kicking the beam with the dipolar kick at increasingly higher strengths and acquiring TbT BPM data one can sample the large-amplitude betatron motion and fit it in the time domain to extract the fundamental frequency, which is the tune. The differences from the initial tune, $\Delta\nu$ are then evaluated.

This procedure was performed for each RCDS sextupole configuration in the 3 working points, specializing to the horizontal tune-shifts due to horizontal kicks. The results are shown by Fig.~\ref{fig:adts}. The black-curve is the expected tune-shifts in for the nominal sextupole strengths. It was calculated in the computer model at each working point. The remaining curves were measured for each RCDS-optimized sextupole configuration. The common feature among the curves of the best performing configurations (run 2 for WP1, run 1 for WP2 and the optimized curve for WP3) seems to be the consistent departure from the nominal model curve. This trend for tune-shifts seem to be beneficial for the actual machine.
\begin{figure}
    \includegraphics[width=\columnwidth]{Images/opt_configs_dtunes.pdf}
    \caption[Horizontal tune-shifts vs. horizontal betatron actions for the RCDS solutions and for the computer model in WPs 1, 2, and 3.]{Horizontal tune-shifts vs. horizontal betatron actions for the RCDS solutions and for the computer model in WPs 1, 2, and 3.}
    \label{fig:adts}
\end{figure}
