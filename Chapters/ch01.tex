\chapter{Introduction}
This dissertation describes the optimization work performed at the SIRIUS storage ring with the objective to improve the ring's Dynamic Aperture and injection efficiency.
 This chapter is dedicated to introducing synchrotron light sources, the SIRIUS light source facility and the problem adressed throughout this master's project.

This dissertation is arranged as follows:
\begin{itemize}
    \item Chapter 1 introduces syncrhotron light sources and the SIRIUS project
    \item Chapter 2 introduces the theoretical and scientific background of the dynamics of particles in particle accelerators
    \item Chapter 3 reviews optimization algorithms, introduces online optimization and the Robust Conjugate Direction Search (RCDS) algorithm and presents an overview of its applications
    \item Chapter 4 describes the experimental methods and experiments setup at the SIRIUS storage ring, as well as the results and analysis.
\end{itemize}

\section{Storage ring-based synchrotron light sources}

Synchrotron ratiation (SR) refers to electromagnetic radiation emitted by charged relativistic particles upon acceleration. The phenomenon was theoretically formalized in the early 1900's by Wiechart and Larmor and first observed experimentally at a General Electric's $70~\unit{Mev}$ synchrotron\footnote{The term ``synchrotron" refers to the accelerator  technology based on the synchonicity of the charged particles period of revolution and the frequency of the electromagnetic fields exerting work on it to achieve and mantain high-energies.} accelerator. It was named \textit{synchrotron radiation} for this reason.

Syncrhotron radiation was in fact a negative side-effect of achievieng high-energy particles for studying nuclear and particle physics. But since it is emitted in a narrow angular aperture and can cover a big spectrum from infrared to hard X-rays, its potential for imaging techiniques in condensed matter physics, materials science, molecular biology and chemistry was soon realized.

The community interested in SR first revolved around big high-energy and nuclear physics machines. Those were the so-called 1st generation of synchrotron light sources, in which SR is obtained parasistically. The 2nd generation consists on machines built with the specific goal or being sources of SR. In thses machines, the radiation was emitted when the electrons beams had their trajectories bent at the dipole magnets. In the 1990's, there was a rise in the application of \textit{insertion devices} (IDs) such as \textit{wigglers} and \textit{undulators}. These components were added to straight sections in the machines to introduce additional transverse accelerations for the generation of SR. The 4th generation of light sources was ignaugurated with the start of the comisioning of the MAX IV machine, in Lund, Sweden. The main advances with respect to the previous generatioin is the reduction by one to two orders of magnitude in the figure of merit for SR sources: the emmittance.

Low emmittance is important for achieving a high-brightnes. Brightness is defined by
\begin{equation}
    B(\omega) = \frac{F(\omega)}{\Omega_{xx^\prime}\Omega_{yy^\prime}\frac{\Delta \omega}{\omega}}
\end{equation}
where $F(\omega)$ is the photon flux at energy $E=\hbar\omega$, $\Omega_{uu^\prime}$ is the the $(u,u^\prime)$ photon phase-space volumes, which depends on both the electron-beam and photon-beam distribution; and $\Delta\omega/\omega$ is the frquency bandwith. High-brightness is achieved by achieving higher electron-beam current to yield higher fluxes, while minimizing the emittances.

In a synchrotron light source facility\todo{work on it}
\begin{itemize}
    \item linear accelerator: comprising
    \item booster ring:
    \item transport lines:
    \item storage ring:
    \item experimental hall and beamlines:
\end{itemize}
Synchrotron storage rings store ultra-relativistic electrons beams close to a reference design orbit. The orbit is determined by the deflection magnets strenthgs, the dipoles. The focusing towards the closed orbit is provided by gradient fields, mostly coming from quadrupole magnets at SIRIUS. Since focusing and deflection depends on the beam energy, small energy deviations in the beam's energy can lead to differential focusing. Using an analogy from geometric optics, the beam's focusing at the lens (quadrupoles) depends on its color (energy). To correct this chromatic aberration, ``glasses'' are necessary. They introduce geometric aberations responsible for uniform, energy-independent focusing.


\section{The SIRIUS project}
SIRIUS is the 4th generation storage ring-based synchrotron light source. It was designed, built and it operated by the Brazilian Synchrotron Light Laboratory, at the campus of the Center of Resarch in Energy and Materials (CNPEM), at Campinas, Brazil. At the time of writing, SIRIUS in one of the three machines of its kind operating in the world.

SIRIUS has finished commissioning in 2022 and since 2023 is receiving its first users. It is currently operating for user's beam with a $100~\unit{mA}$ current, but is designed to achieve $350~\unit{mA}$ when the system of superconducting radio-frequency cavities, as well as the higher-oder-harmonic cavity are installed.

\section{This dissertation problem}
    The persuit of low-emmittances and high-brightnes pushed the accelerator community towards the 4th generation of storage rings. In such machines, there is strong focusing provided by quadrupole magnets which also require strong sextupolar fields needed for compensating the dependence of focusing on the beam-energy (the so-called chromaticity). Since sextupole provide nonlinear fields, the dyamics in 4th generation storage rings has become increasingly nonlinear.

    Quasi-periodic nonlinear dynamics subject to the smallest perturbations, such as small fields deviations arising from rotation, alignement or excitation errors, can become unstable at large oscillation amplitudes. The instabilities result in limitations to large transverse oscillation amplitudes the machine can support. This amplitude is known as the Dynamic Aperture of the ring.

    In ordinary operation condition, the equillibrium beam distribution is much smaller than the DA. One situation in which the DA is important for operation is during the injection process. The beam is extracted from the booster accelerator and enters the storage ring after passing through a transport line. It is then deflected by pulsed nonlinear magnets to make it parallel to the storage ring, but with a horizontal offset of approximately $x=-8~\unit{mm}$. The DA thus impacts over the injection efficiency, and it is crucial for establishing repeatability in the efficiency, which is important for the top-up operation mode, in which beam is injected continuously throughout the operation with the objective to keep the current approximately constant.

    The choice of placement, symmetry and strength of sextupoles was chosen based on optimization of the simulated dynamic aperture and beam-lifetime in the machine computer model. The average performance of the configurations in the presence of several error sources was optimized and the obtained optimzed lattice was then implemented in the machine during the comissioning phase. The real machine, is basically a realization of such error configurations, and renders the machine a certain performance.

    Prior to the optimization work, the Dynamic Aperture was measured to be \todo{get data}, which rendered an average of \todo{get data} injection efficiency. The main difficulty was the typycal fluctuations in the effciency \todo{get data}.

    Since the lattice configuration implemented is in principle close to the optimum configuration, it is reasonable to assume that small nudges and adjustments on the sextupoles strengths could accomodate the lattice to the physically realized errors distribution, improving the nonlinear dynamics performance, the DA and ultimately, the injection efficiency and its stability.
