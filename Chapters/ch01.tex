\chapter{Introduction}
This dissertation presents the optimization work performed on the SIRIUS storage ring sextupole mangets with the objective to improve the ring's Dynamic Aperture and injection efficiency. The text is organized as follows:
\begin{itemize}
    \item Chapter 1 introduces syncrhotron light sources and the SIRIUS project
    \item Chapter 2 introduces the theoretical and scientific background of dynamics of particles in particle accelerators. Nonlinear dyanamics is presented and its consequences are presented. Particular attention is drawn to the Dynamic Aperture.
    \item Chapter 3 reviews optimization algorithms, introduces online optimization in accelerators and the Robust Conjugate Direction Search (RCDS) algorithm.
    \item Chapter 4 describes the experimental methods and experiments carried out at the SIRIUS storage ring.
    \item Chapter 5 concludes this dissertation presenting some final remarks.
\end{itemize}

\section{Storage ring-based synchrotron light sources}

Synchrotron radiation (SR) refers to electromagnetic radiation emitted by charged relativistic particles as a result for being accelerated. The emitted intensity is the largest along the direction perpendicular to the that of the acceleration. In a circular accelerator, this implies the emission occurs along the curved trajectory's tangent.

The phenomenon was theoretically predicted in the early 1900's when Liénard  and Wiechart calculated the retarded potentials--introduced by L. Lorenz--for point particles. Liénard calculated the energy lost by the electron due to the radiation emission and recovered Larmor's famous result \todo{Where's Larmor in all of this?}. Synchrotron light was firts observed experimentally at a General Electric's $70~\unit{Mev}$ synchrotron\footnote{The term ``synchrotron" refers to the accelerator  technology based on the synchonicity between the charged particles period of revolution and the frequency of the electromagnetic fields exerting work on it to achieve and mantain high-energies.} accelerator. It was named \textit{synchrotron radiation} for this reason.

For nuclear and particle physicists, whom the first accelerators served, SR was in fact a detrimental side-effect of achievieng high-energies. But since the light is strongly colimated and covers a broad spectrum, from infrared to hard X-rays, its potential for imaging techiniques in condensed matter physics, materials science, molecular biology and chemistry was soon realized.

The community interested in SR for these purposes first revolved around big high-energy and nuclear physics machines and obtained SR parasistically from them, in the early 1960's. These are the so-called first-generation synchrotron light sources. With the increase of experiments using SR, accelerators were built for this specific purpose, which ignaugurated the second-generation. In thses machines, light was produced at the bending magnets.

A figure of merit measuring the quality of a SR source is its \textit{brightness}, defined as the photon flux per unit area and per unit solid angle at the source
\begin{equation}
    B(\omega) = \frac{F(\omega)}{\Omega_{xx^\prime}\Omega_{yy^\prime}\Delta \omega/\omega},
\end{equation}
where $F(\omega)$ is the photon flux at energy $E=\hbar\omega$, $\Omega_{uu^\prime}$ is the the $(u,u^\prime)$ photon phase-space volumes, which depends on both the electron beam and photon beam distributions; $\Delta\omega/\omega$ is the frquency bandwith.
The growing community based on spectroscopy and crystallography experiments  soon started to require  maximum flux within small phase space volume  (brightness) to improve the spectral resolution and match  the incident beam to the small crystal sizes

% High-brightness is achieved by achieving higher electron-beam current to yield higher fluxes, while minimizing the electron beam emittances--the beam's average phase space volume.

In the 1990's, the introduction of \textit{insertion devices} (IDs) such as \textit{wigglers} and \textit{undulators} characterized the third-generation machines. These devices consists on arrays of alternating dipolar fields which introcue additional transverse accelerations to the beam for the light production. The additional radiation output increased radiative damping reducing emmittance and increasing brightness. IDs also allowed for control of radiation energy and polarization.


The 4th generation of SR sources was ignaugurated with the start of the comisioning of the MAX IV machine, in Lund, Sweden. The main advances with respect to the previous generatioin is the reduction by one to two orders of magnitude in the emmittance, which was made possibile due to the recent technological advances. Soon after, an upgrade of the European Synchrotron Facility (ESRF) machine redered it the second 4th generation machine in the world (when?). SIRIUS, in Campinas, is the third in the world, the first in the global-south.

\todo[inline]{what about FEL's? Why are you always making sure to clarify youre talking about storage rings?}

Usually, the accelerators systems of a synchrotron light source facility consists on the following devices
\begin{itemize}
    \item electron gun (eGun): from which electrons are obtained. At SIRIUS, a cathode is used, and the electrons are ejected by thermoionic emission, at $90~\unit{keV}$ energy.
    \item linear accelerator (LINAC): consisting on a array of cavities along whcih RF fields propagate with increasing group velocity, carrying the electron along with it. SIRIUS' LINAC accelerates electrons to about $150~\unit{MeV}$.
    \item booster ring: syncrhotron accelerator in which the electron energy is ramped to the storage ring's nominal operating energy of $3~\unit{GeV}$.
    \item transport lines: along which the electron is transported from the linac to booster (LTB) and from the booster to the storage ring (LTS).
    \item storage ring: where ultra-relativistic electrons are kept stable during hours, oscillating around a closed orbit.
\end{itemize}
Synchrotron storage rings store ultra-relativistic electrons beams close to a reference design orbit. The orbit is determined by the deflection magnets strenthgs, the dipoles. The focusing towards the closed orbit is provided by gradient fields, mostly coming from quadrupole magnets at SIRIUS. Since focusing and deflection depends on the beam energy, small energy deviations in the beam's energy can lead to differential focusing. Using an analogy from geometric optics, the beam's focusing at the lens (quadrupoles) depends on its color (energy). To correct this chromatic aberration, ``glasses'' are necessary. They introduce geometric aberations responsible for uniform, energy-independent focusing.
\missingfigure{mangets and devices}

The accelerators  are kept within concrete tunnels for radiation safety. Tangentially to the nominal orbits, optical beamlines direct the SR photon beams to the experimental cabins where the experiments in condensed matter physics, chemistry, molecular and celular biology are carried out. SIRIUS has X experimenta beamlines.
\missingfigure{light source layout}




\section{The SIRIUS project}
SIRIUS is the 4th generation storage ring-based synchrotron light source. It was designed, built and it operated by the Brazilian Synchrotron Light Laboratory, at the campus of the Center of Resarch in Energy and Materials (CNPEM), at Campinas, Brazil. At the time of writing, SIRIUS in one of the three machines of its kind operating in the world.

SIRIUS has finished commissioning in 2022 and since 2023 is receiving its first users. It is currently operating for user's beam with a $100~\unit{mA}$ current, but is designed to achieve $350~\unit{mA}$ when the system of superconducting radio-frequency cavities, as well as the higher-oder-harmonic cavity are installed.

\section{This dissertation problem}
    The persuit of low-emmittances and high-brightnes pushed the accelerator community towards the 4th generation of storage rings. The particular arrangement of bending magnets which allowed for such low-emmittances also requitres intese gradient fields provided by quadrupole magnets. Strong focusing, in turn, demands strong sextupolar fields for compensating chromatic effects, such as focusing dependence on the beam's small energy fluctuations. Since sextupole provide nonlinear fields, the dyamics in 4th generation storage rings has become increasingly nonlinear.

    A quasi-periodic nonlinear dynamics subject to the smallest perturbations, such as small fields errors arising from rotation, alignement or excitation errors, can become unstable at large oscillation amplitudes. The instabilities result in limitations to large transverse oscillation amplitudes the machine can support. This amplitude is known as the Dynamic Aperture of the ring.

    In ordinary operation condition, the equillibrium beam distribution is much smaller than the DA. One situation in which the DA is important for operation is during the injection process. The beam is extracted from the booster accelerator and enters the storage ring after passing through a transport line. It is then deflected by pulsed nonlinear magnets to make it parallel to the storage ring, but with a horizontal offset of approximately $x=-8~\unit{mm}$. If the DA is smaller than that, it limits the injection efficiency for beam accumulation.
    %  and it is crucial for establishing repeatability in the efficiency, which is important for the top-up operation mode, in which beam is injected continuously throughout the operation with the objective to keep the current approximately constant.

    The choice of placement, symmetry and strength of sextupoles was chosen based on the optimization of the simulated dynamic aperture and beam-lifetime in the machine computer model. The average performance of the configurations in the presence of several magnets errors (simulating the expected errors in the real machine)  was optimized and the obtained optimzed lattice was then implemented in the machine during the comissioning phase. The real machine, is basically a realization of such error configurations, and renders the machine a certain performance.

    Prior to the optimization work, the Dynamic Aperture was measured to be \todo{get data}, which rendered an average of \todo{get data} injection efficiency. The main difficulty was the typycal fluctuations in the effciency \todo{get data}.

    Since the lattice configuration implemented is in principle close to the optimum configuration, it is reasonable to assume that small nudges and adjustments on the sextupoles strengths could accomodate the lattice to the physically realized errors distribution, improving the nonlinear dynamics performance, the DA and ultimately, the injection efficiency and its stability. Online optimization consists on choosing computer-automated direct search strategies to seek the optimum sextupole configurations rendering the larges dynamic aperture.
