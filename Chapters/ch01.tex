\chapter{Introduction}
This dissertation concerns the work performed on the SIRIUS storage ring sextupole mangets with the objective to optimize the ring's Dynamic Aperture and injection efficiency. The text is organized as follows:
\begin{itemize}
    \item The present chapter introduces synchrotron light sources, the SIRIUS project, the main components and subsystems found in electron storage rings and the problem trackled during the execution of the masters project;
    \item Chapter 2 introduces the theoretical and scientific background on the dynamics of particles in accelerators. The chapter goal is to define the Dynamic Aperture;
    \item Chapter 3 introduces and justifies online optimization in accelerators and the Robust Conjugate Direction Search (RCDS) algorithm;
    \item Chapter 4 describes the experimental methods and setup for the online optmization experiments and the results achieved;
    \item Chapter 5 concludes this dissertation presenting the final remarks.
\end{itemize}

\section{Storage ring-based synchrotron light sources}
Synchrotron radiation (SR) is the electromagnetic radiation emitted by charged relativistic particles when accelerated perpendicularly to their motion. The phenomenon was theoretically predicted in the early 1900s when Liénard and Wiechert calculated the retarded potentials for point particles. The first experimental observation occurred at General Electric's synchrotron accelerator, justifying the adoption of the term "synchrotron" in its name \cite{wiedemann_particle_2015}.

Synchrotron light is extermely colimated and has a broad spectral distribution, covering from infrared to hard X-rays. These properties make synchrotron light ideal for imaging experiments in crystallography and spectroscopy in a wide variaty of scientific disciplines.

Modern synchrotron light sources primarily rely on two particle acceleration technologies: free-electron lasers and electron storage rings. Here, we focus on storage ring-based synchrotron light source facilities. In these facilities, ultra-relativistic electron beams are stored for extended periods within a chamber in ultra-high vacuum to produce synchrotron light. The beams are maintained in stable orbits by the fields of an array of magnets--the lattice--which provide both bending, focusing and trajectory-correction fields. The beam is also periodically influenced by radiofrequency cavities, which replenish the energy radiated away in the form of light.

The main figure of merit for the quality of a SR source is the \textit{brightness}\cite{huang_brightness_2013}, defined as the photon flux in six-dimensional phase space \cite{hettel_challenges_2014}:
\begin{equation}
    B(\omega) = \frac{1}{\Delta \omega/\omega}\frac{F(\omega)}{\Sigma_{x}(\omega)\Sigma_{y}(\omega)},
\end{equation}
where $F(\omega)$ is the photon flux at energy $E=\hbar\omega$, $\Sigma_{u}$ is the photon beam volume in the the $u=x,y$ phase space, and $\Delta\omega/\omega$ is the frquency bandwith, which is typically about $0.1\%$. The photon phase space volume depends on the convolution of the electron beam distribution with the distribution of the photons emmited by a single electron. The latter depends on the photon energy and the emission process, while the former is related to the average phase space volume of the electron beam: the \textit{emmitance}, which depends on the magnetic lattice and has units of the transverse phase space areas (length times angle). Increasing brightness depends on maximizing the photon flux, reducing the electron beam emittances and optimizing the matching between photon and electron beams for maximizing their distributions convolution \cite{wiedemann_particle_2015}.

Synchrotron light sources can be classified based on their brightness/emmitances. In the early 1960s, the community interested in synchrotron radiation (SR) for imaging experiments obtained SR parasitically from high-energy and nuclear physics machines such as DESY and DORIS in Germany and ADA in Italy \cite{simoulin_synchrotron_2016}, marking the era of first-generation synchrotron light sources \cite{liu_towards_2017}. The second-generation machines emerged in the 1980s and consisted on machines designed exclusively for SR production, such as BESSY, DORIS II and III, ELSA in Germany, SUPERACO in France, MAX I in Sweden \cite{simoulin_synchrotron_2016} and UVX in Brazil.

The 1990s saw a growing demand for higher brightness, leading to the development of third-generation machines \cite{liu_towards_2017}. These machines introduced insertion devices such as wigglers and undulators, significantly enhancing brightness by increasing radiative damping. Additionally, these devices allowed precise control over radiation energy and polarization. Typical emmitance for third-generation machines is of the order of units to tens of $\unit{nm}~\unit{rad}$. Most of currently operating machines pertain to the third-generation, such as ALBA in Spain, ESRF and SOLEIL in France, Diamond in the United Kindom and ELETTRA, in Italy \cite{simoulin_synchrotron_2016}.

The era of the fourth-generation of storage rings (4GSR) commenced with the commissioning of the MAX-IV machine in Lund, Sweden, in 2015 \cite{liu_towards_2017,hettel_challenges_2014}. Fourth-generation machines achieved a notable reduction in emittance, reaching sub-$\unit{nm}~\unit{rad}$ values thanks to recent technological advancements in computer simulations, vaccum technology, machining and mechanical alignment \cite{hettel_challenges_2014,liu_towards_2017}. Following MAX-IV, an upgrade of the ESRF facility, the ESRF-EBS, and the launch of SIRIUS in Campinas marked significant milestones for the fourth-generation. Today, several 4GSR projects are being planned, designed and constructed.

\section{The SIRIUS project}
SIRIUS is a 4GSR synchrotron light source. It was designed, built, and is operated by the Brazilian Synchrotron Light Laboratory (LNLS), on the campus of the Center of Research in Energy and Materials (CNPEM), in Campinas, Brazil. The storage ring operating energy is $3~\unit{GeV}$, and the ring has $518~\unit{m}$ in circumference. The natural emmitance of the lattice is $250~\unit{pm}~\unit{rad}$ and it is expected to reach $150~\unit{pm}~\unit{rad}$ with the installation of its definitive insertion devices \cite{liu_synchrotron_2019}\footnote{SIRIUS is currently operating with provisional insertion devices for providing light to the first users and allowing scientific commissioning of the beamlines}.

SIRIUS succeeded the first synchrotron light source in Brazil, the UVX machine, which opened to users in 1997 and served the community until its shutdown, in the beginning of SIRIUS commissioning, in August 2019\footnote{The UVX project led to the creation of LNLS, which marked a new model for scientific research in Brazil, based on social organizations under contracts with the Ministry of Science Technology and Innovations. LNLS paved the way for national labs (NL), including labs on biosciences (LNBio), nanotechnology (LNNano), and bio-renewables (LNBR), which are also located at the CNPEM campus.}\cite{liu_synchrotron_2019}. The project started in 2009, initially planned and designed as a third-generation machine. By 2012, it evolved into the project of a fourth-generation machine\cite{liu_synchrotron_2019}. Cunstruction was finished in 2018, when the LINAC and Booster commissioning started. In november 2019 the first beam was stored in the storage ring.

SIRIUS finished its Phase-0 commissioning in 2022 and since March 2023 is receiving its first external users. At the time of this writing, it has 6 operating beamlines, 4 beamlines in commissioning and 4 under construction and installation. It is currently operating for user's beam with a $100~\unit{mA}$ current, with frequent beam injections throughout the day, a scheme known as ``top-up'' mode. SIRIUS is expected to achieve $350~\unit{mA}$ current when the system of two superconducting radiofrequency cavities is installed \cite{liu_status_2022,liu_status_2023}.
\todo[inline]{more info on the phases. which phase are we in?}

Presently, SIRIUS stands as the most complex scientific infrastructure ever consturcted in Brazil, with the ambitious goal of positioning the country at the forefront of global leadership in synchrotron light sources. This state-of-the-art synchrotron was meticulously designed to shine as the brightest in its energy category, and has the capacity to host up to 40 beamlines. As of the time of this writing, SIRIUS holds the distinction of being the sole fourth-generation synchrotron radiation source in the southern hemisphere and one of merely three 4GSRs in operation across the globe (still true?).
\section{Subsystems and components of a storage ring-based light source facility}
Typical systems comprising a storage ring synchrotron light source facility include:
\begin{itemize}
    \item an injection system: including the electrons source, beam transport lines, the linear accelerator and the booster circular accelerator. At SIRIUS, the linear accelerator provides the booster ring with a $150~\unit{MeV}$ beam. The booster further ramps the beam energy up to $3~\unit{GeV}$, which is the storage ring operation energy;
    \item storage ring: where ultra-relativistic electrons are kept stable for hours within the vaccum-chamber, oscillating about a closed orbit for the production of synchrotron light;
    \item beamlines which steer the photon beams towards the experimental cabins where samples are placed for the experiments based on light-matter interaction, such as spectroscopy, crystallography, tomography and others.
\end{itemize}
A schematic view of the SIRIUS building is shown in Fig.~\ref{fig:sirius_layout}.
\begin{figure}[t]
    \centering
    \includegraphics[width=0.8\textwidth]{Images/sirius_facility.png}
    \caption{Schematic view of the SIRIUS installations. 1) Linear accelerator (LINAC); 2) Concrete tunnel housing the booster accelerator and the storage ring; 3) storage ring; 4) beamlines. From \href{https://lnls.cnpem.br/sirius/como-funciona-o-sirius/}{LNLS website}.}
    \label{fig:sirius_layout}
\end{figure}


In a synchrotron storage ring, ultra-relativistic electron beams are stored in proximity to a reference design orbit. This orbit is determined by the strengths of the deflection magnets, the dipoles, and the operation energy of the beams. A pure dipole provides an uniform and homogeneous magnetic field perpendicular to the floor. To define a closed orbit, the overall bending angle provided by the dipoles along the entire ring must equal $2\pi$ radians. The field profile of a dipole magnet is depicted in the left side sketch of Fig.~\ref{fig:magnets_fields}. Imagining a beam directed inward the screen, the trajectory will be bent to the right.

To maintain electrons in close proximity to the reference orbit, focusing is achieved through gradient fields, primarily generated by quadrupole magnets at SIRIUS. The strength of such fields increase linearly with deviations from the closed orbit, which lies in the magnet's center. Gradient fields effectively act as spring forces. The mangets poles and the field profile of a quadrupole magnet are depicted in the center sketch of Fig.~\ref{fig:magnets_fields}.

Focusing and deflection are energy-dependent, which means small deviations from the nominal operating energy can result in differential focusing. Drawing an analogy from geometric optics, the beam's focusing behavior at the "lens" (quadrupoles) depends on its "color" (energy). To correct for these chromatic aberrations, the use of "glasses" becomes necessary. In the context of accelerators, sextupole fields serve as these corrective lenses. They introduce geometric aberrations to counteract the chromatic ones, resulting in approximately uniform, energy-independent focusing. The mangets poles and the field profile of a sextupole magnet are depicted in the right sketch of Fig.~\ref{fig:magnets_fields}.
\begin{figure}[htb]
    \includegraphics[width=\textwidth]{Images/magnets.pdf}
    \caption{Schematic representation of the magnets comprising SIRIUS lattice and their fields profile. From left to right: dipole magnet, quadrupole magnet and sextupole magnet.}
    \label{fig:magnets_fields}
\end{figure}

Besides dipoles, quadrupoles and sextupoles, additional dipole actuators magnets for orbit/trajectory correction and pulsed magnets for beam extraction and injection can also be found in SIRIUS.

The beam is also periodically subject to longitudinal time-dependent electromagnetic radiofrequency fields which do work on the beam to replenish its energy. The goal is to compensate for the energy carried away in the form of synchrotron light.

\section{The problem addresed in this work}
The pursuit of low emittances and high brightness has driven the accelerator community toward the fourth-generation of storage rings. Achieving such low emittances was possible because of series of technological advances which enabled the use of the multi-bend achromat (MBA) lattice\cite{liu_towards_2017,hettel_challenges_2014}. MBA lattices require intense gradient fields provided by quadrupole magnets, which, in turn, demands the presence of strong sextupolar fields to compensate for chromatic effects. Since sextupoles provide nonlinear fields, the dynamics in fourth-generation storage rings has become increasingly nonlinear\cite{liu_towards_2017}.

A quasi-periodic nonlinear dynamics, when subjected to even the slightest perturbations—such as small field errors stemming from rotation, alignment, or fields excitation errors—can potentially become unstable at large oscillation amplitudes. These instabilities impose constraints on the maximum transverse oscillation amplitudes that the machine can accommodate. This specific amplitude below which motion stable is referred to as the Dynamic Aperture of the ring.

Under normal operating conditions, the equilibrium beam size and typical oscillation amplitudes are considerably smaller than the Dynamic Aperture (DA), and the dynamics can be well studied and analyzed using a linear approximation theory, without worrying about the DA. However, there are specific scenarios where the DA becomes crucial for the operation, notably during the injection process.

During injection into the storage ring for beam accumulation, the beam is extracted from the booster accelerator and guided toward the storage ring through a transport line. Subsequently, it is deflected by pulsed nonlinear magnet so it becomes almost parallel to the storage ring tangent direction, albeit with a horizontal offset of approximately $x=-8~\unit{mm}$ \cite{liu_injection_2016}. If the DA is smaller than this offset, it imposes limitations on the injection efficiency.

The placement, symmetry, and strength of sextupoles magnets--the nonlinear lattice-- were determined through a multi-objective optimization process , primarily focusing on improving the simulated dynamic aperture and beam lifetime of the machine's computer model\cite{de_sa_optimization_2016, dester_energy_2017}. This optimization work considered the average performance of the lattice configurations while accounting for various magnet errors that simulate the expected errors in the actual machine \cite{de_sa_optimization_2016}.

Several models, with several errors distribution among the magnets were generated, and the DA and lifetime for a given lattice configuration was calculated by simulating the electron beams motion for several turns (tracking simulations). The final figure of merit for a lattice consisted on the average DA and lifetime it provided to the ensemble of machines. The best-performing machine lattice found during this process was adopted as the nominal lattice which was subsequently implemented during the commissioning phase of the machine \cite{de_sa_optimization_2016}.

However, the operating machine consists on a practical realization of a specific error configuration, which defines the physically realized magnetic lattice and its overall performance. The best-performing lattice on average is not necessarily the optimum lattice for this errors realization.

Assuming the realized lattice closely approximates the optimum setup, i.e. that errors are small, it is reasonable to assume that making minor tweaks and adjustments to the strengths of the sextupoles can adapt the lattice to match the actual distribution of errors in the physical system. This fine-tuning process can result in improved nonlinear dynamics performance, expanding the Dynamic Aperture (DA), and ultimately enhancing injection efficiency and its stability.

Online optimization consists on employing computer-automated search strategies to systematically explore various sextupole configurations with the goal of identifying the one that yields the largest dynamic aperture while not interfering in other machine parameters. We specilize to online optimization using the Robust Conjugate Direction Search algorithm, which has been broadly applied to optimization in accelerators.

It is shown in this dissertation that this optimization process can succesfully improve the dynamics performance. Prior to the optimization work, the Dynamic Aperture was measured to be \todo{get data}, which rendered an average of \todo{get data} injection efficiency. The main difficulty was the typycal fluctuations in the effciency \todo{get data}.

Besides improving the injection efficiency in ordinary operation conditions, it is also interesting to do so in different machine \textit{working points}, with different \textit{tunes}. As chapter 2 shows, if one fixes his or her attention to a specific point of the ring, and measure the beam position in horizontal and vertical planes for consecutive turns, one realizes the motion is a sampled sinusoid. The tunes $\nu_x$ and $\nu_y$ are the fundamental frequencies of such harmonic motion. They are important operation parameters  since they affect over the response of the system in the presence of pertrubations. Tunes close to integer numbers make the dynamics particularly sensitive to perturbations. SIRIUS nominal tunes fractional parts are quite low, and increasing them push the tunes away from integers, reducing the \textit{orbit amplification factors} when the orbit is disturbed. Increasing the tunes can be achieved by actuating wiht the quadrupoles, but doing so brings the machine to a different optics, in which the DA can, and indeed is as we shall see, smaller than the expected value. In different working points, the online optimization is essential to achieve a good DA rendering acceptable injection efficiencies for operation.

In this work we show how we have succesfully optimized the dynamic aperture and injection efficiency in the nominal tunes and in different machine working points. At the time of this writing, the machine is operating with the optics found during one of the experiments carried out during the execution of this project. We show how the increased tunes led to unprecedent orbit stability and how online optimization made possible to achieve high injection efficiencies, which allowed for the operation with these tunes.
