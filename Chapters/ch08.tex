\chapter{Momentum Compaction Factor and the relation between energy deviations and RF frquency changes}
Momentum compaction factor is the quantity characterizing the energy/momentum dependence of revolution time/frequency of orbits.
The path length traversed by a particle reads up to first order reads
\begin{equation}
    \dd \ell = (1 + G(s)x)\dd s
\end{equation}
where $x(s) = x_\beta(s) + \eta(s)\delta$. For $\delta = 0$ we have simply
$$L = \oint (1 + G(s)x_\beta(s))\dd{s}$$
thus the additional length traversed by an off-energy particle reads
\begin{equation}
    \delta\ell = \oint G(s)\eta(s)\delta \dd{s}
\end{equation}
thus we can write
$$\frac{\delta\ell}{L}=\alpha \delta$$
by defining the \textit{momentum compaction factor}:
\begin{equation}
    \alpha = \frac{1}{L}\oint G(s)\eta(s)\dd{s}
\end{equation}
For relativistic electrons, the increase in energy leads to enlargement of orbits with neglible increase of velocity. Thus, the orbit revolution time decreases. This apparently paradoxical result.
$$\frac{\Delta T}{T}=\qty(\alpha-\frac{1}{(v/c)^{2}\gamma^{2}})\delta$$
where $\delta = \Delta E/E$. For $v\to c$, (large $\gamma$), we have
$$\frac{\Delta T}{T}=\alpha \frac{\Delta E}{E}$$
or, equivalently
$$\frac{\Delta f}{f}=-\alpha \frac{\Delta E}{E}$$


For non-relativistic electrons, the increase in energy leads to increase of velocity and the orbit time remains fixed. This is what makes cyclotrons possible.
