\begin{abstract}
    Electron beam accumulation into SIRIUS storage ring occurs in a off-axis scheme, on which the efficiency depends on a sufficiently large dynamic aperture (DA): the region comprising stable transverse oscillations.
    % In the design phase, SIRIUS DA was numerically optimized in the accelerator model using various techniques, and during commissioning, the optimized lattice was implemented in the machine.
     Prior to optimization, SIRIUS DA was sufficiently large for providing good injection efficiency levels, but showed signs it could be further improved upon fine-tuning of sextupole magnet strengths,
    %  which govern the beam nonlinear dynamics
     which determine  the DA.
    %  The tuning process is known as Online Optimization and has proven succesful in improving the DA and injection efficiency of SIRIUS, in agreement with the observed outcomes in other machines worldwide.
     In this master's project, the Robust Conjugate Direction Search Algorithm (RCDS) for online optimization was applied to optimize the DA and the injection efficiency at SIRIUS. Online optimization was also carried out in different working points, with higher fractional tunes. Higher tunes are beneficial for orbit stability, but usually deteriorate the DA, requiring online optimization work to achieve injection efficiencies meeting operation demands. This work shows how online optimization was successful at optimizing the DA in SIRIUS nominal working point as well as in a new working point where orbit stability improvements were observed. The results highlight the effectiveness of this method in 4th-generation storage rings.
\end{abstract}
