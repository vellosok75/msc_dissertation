\begin{abstract}
    Beam accumulation into the SIRIUS storage ring occurs in the off-axis scheme, for which the efficiency depends on a sufficiently large dynamic aperture (DA) - the region comprising stable transverse oscillations. In the design phase, SIRIUS DA was numerically optimized in the accelerator model using various techniques, and during commissioning, the optimized lattice was implemented in the machine. Recent measurements indicate that SIRIUS DA, although sufficiently large for good injection efficiency, can be increased further upon fine-tuning of sextupole magnet strengths, which govern the beam nonlinear dynamics and determine the DA. Additionally, growing interest in operating in different optics working points, where the DA is usually deteriorated, requires online optimization work to achieve acceptable injection efficiencies for operation. In this master's project, the student studied and implemented in Python the Robust Conjugate Direction Search Algorithm (RCDS) and carried out online optimization experiments to improve the DA and the injection efficiency. The results highlight the effectiveness of online optimization in 4th-generation storage rings.
\end{abstract}
