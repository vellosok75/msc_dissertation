
\begin{abstract}[name=Agradecimentos]

    \phantom{----}Primeiramente, agradeço à minha orientadora, Liu. Sua experiência e seu direcionamento foram fundamentais para o sucesso deste projeto. Agradeço por todo apoio, incentivo, paciência e pela oportunidade de participar deste projeto e de trabalhar junto a pessoas tão experientes e inspiradoras num ambiente tão estimulate e rico.

    Agradeço também ao Professor Rubens pela coorientação. Obrigado pelo interesse constante no trabalho e por se fazer sempre presente, além de sempre ter feito  questão de oferecer belas palavras de congratulação e reconhecimento das pequenas conquistas.

    Agradeço aos meus pais e familiares. Obrigado pelo constante apoio, compreensão e paciência, especialmente pelas minhas ausências. Também agradeço à minha companheira, cujo amor, carinho e apoio foram minhas principais fontes de força e conforto. Obrigado, Juba.

    Sou grato aos funcionários do CNPEM e do LNLS, pelo esforço conjunto para erquer um ambiente único para a ciência e tecnologia no Brasil. Aos colegas do grupo de Física de Aceleradores (FAC) do LNLS, sou extremamente grato pela colaboração e momentos de descontração compartilhados durante todo o período de trabalho. Obrigado Fernando, Murilo, Ximenes e Gabriel pelas discussões valiosas e sempre muito produtivas, pela leitura desta dissertação, e por toda a colaboração durante este projeto. Aprendi muito com vocês e espero ainda aprender muito mais. Agradeço também ao Gustavo, Thales, Vitor, Raphael, Filipe e Jucelio pelos momentos que compartilhamos em 2022 e 2023.

    Sou grato ao IFGW pela infraestrutura proporcionada durante o mestrado e pela oportunidade de ter acesso aos cursos avançados e contato com os  excelente professores. Em especial, agradeço ao professor Amir Caldeira por suas excelentes aulas de mecânica quântica e ao professor Marcos Aguiar pelo excelente curso de mecânica avançada. Agradeço também aos professores Diego Muraca, José Brum e Alexandre Fonseca pela orientação durante minha atuação no programa de estágio docente, e aos funcionários do programa de pós-graduação, em especial à Cristina, pelo atendimento proativo e atencioso.

    Também gostaria de expressar minha gratidão aos professores e pesquisadores que contribuíram com críticas e sugestões construtivas para este trabalho. Além de minha orientadora, coorientador e colegas da FAC, agradeço pelo feedback dos professores José Brum, Eduardo Granado e Júlio Criginski, durante o exame de qualificação, e pelo feedback de Túlio Rocha e Eduardo Miqueles, durante no seminário de pré-requisito.

    Por fim, expresso minha gratidão às agências de fomento CAPES e FAPESP pelo apoio financeiro via processos CAPES 88882.328986/2010-01 e FAPESP 2022/04162-4.
    % \vspace{1.5cm}

% \noindent{\LARGE \textbf{Acknowledgements}}\\
%     \vspace*{1.5cm}

    I extend my gratitude to Dr. Xiabiao Huang for his unwavering availability and collaborative spirit. I am truly thankful for the opportunity to have participated in the online optimization experiments carried out in SIRIUS during  Dr. Huang visit in February 2023. Furthermore, I am deeply appreciative of Dr. Huang for recommending our work to be presented at the Workshop on Optics Tuning and Corrections for Future Colliders.
    % I would also like to express my gratitude to Jacqueline Keintzel, Rogelio Thomas Garcia, and Frank Zimmerman for their warm hospitality during my attendance at the workshop and CNPEM and FAPESP for making the making the trip possible.
\end{abstract}
